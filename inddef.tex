\documentclass[11pt,a4paper]{ltjsarticle} % lualatex
%
\usepackage{fontspec}
%
\usepackage{amsmath,amssymb} % 数式用
\usepackage{amsthm} % 定理環境
\usepackage{braket} % カッコ
\usepackage{prettyref} % 相互参照
%
%%%%======prettyref===========%%%%%%%%%
%
\newrefformat{eq}{式{\eqref{#1}}}
\newrefformat{thm}{定理{\ref{#1}}}
\newrefformat{ex}{例{\ref{#1}}}
\newrefformat{lemma}{補題{\ref{#1}}}
\newrefformat{que}{問{\ref{#1}}}
%
%%%%%%======biblatex=============%%%%%%%%%%
%
\usepackage[% 
backend=biber,
url=true,
doi=false,
eprint=false,
isbn=true,
style=phys
]{biblatex} % use "biblatex"
\DeclareSortingTemplate{mysorting}{
  \sort{\citeorder}
  \sort{\field{sotname} \field{author} \field{translator} \field{editor}}
  \sort{\field{year}}
  \sort{\field{title}}
}
%
\DeclareFieldFormat{title}{\mkbibacro{#1}}
\ExecuteBibliographyOptions{sorting=mysorting}
%
\addbibresource{bib/youryou.bib}
%
%%%%%%%%%%%%%%========hyperref============%%%%%%%%%%
%
\usepackage[unicode]{hyperref} % hyperlink
\hypersetup{% setting hyperref
  bookmarksnumbered=true,
  bookmarksopen=true,
  bookmarkstype=toc,
  pdfborder={0 0 0},
  colorlinks=true
}
%
%%%%%%==========定理環境のカスタマイズ===============%%%%%%%%%
%
\newtheoremstyle{mystyle} % 定理環境のスタイルを作成
{} % space above
{} % space below
{\normalfont} % body font 
{} % indent amount
{\bfseries} % theorem head font
{} % punctuation after theorem head
{10pt} % space after theorem head
{\thmname{#1}\thmnumber{#2}\thmnote{\textbf{\hspace{1pt}(#3)}}} % theorem head spec
\theoremstyle{mystyle} % 作成したスタイルを使用
%
\newtheorem{thm}{定理} % \begin{thm} で定理を書く
\newtheorem{lemma}{補題} % \begin{lemma} で補題
\newtheorem*{coro}{系}  % \begin{coro} で系
\newtheorem{que}{問} % \begin{que} で問
\newtheorem*{ans}{[解答]} % \begin{ans} で解答
\newtheorem*{ans2}{[別解]} % \begin{ans2} で別解
%
%
%%%%%================proof環境のカスタマイズ========================%%%%%%%%%%%
\makeatletter
\renewenvironment{proof}[1][\proofname]{\par
  \pushQED{\qed}%
  \normalfont \topsep6\p@\@plus6\p@\relax
  \trivlist
\item[\hskip\labelsep
  \itshape
  %    #1\@addpunct{.}]\ignorespaces% DELETED
  #1]\ignorespaces% ADDED
  }{%
  \popQED\endtrivlist\@endpefalse
}
\makeatother
%                       ドットが消える
\renewcommand{\proofname}{\textbf{[証明]}}
%
\renewcommand{\labelenumi}{(\arabic{enumi})}
%
%%%%%%%%%%%%==========タイトル===============%%%%%
\title{漸化式の解を推測するアレがアレな件}
\author{野口 匠}
\date{\today}
\begin{document}
%
\maketitle

本稿では,しばらくの間は初等・中等教育の慣例に合わせ,自然数は正の整数のことであるとし,
自然数全体の集合を$\mathbb{N}$と表す.「$n \in \mathbb{N}$」は
「$n$は自然数である」ことを表す.
また,数学的帰納法のことをしばしば帰納法と略記する.
ここでいう数学的帰納法とは,
自然数$n$に関する条件$P(n)$について,
\begin{enumerate}
  \item $P(1)$が成り立つ.
  \item 各$k$について,$P(k)$が成り立つならば$P(k+1)$が成り立つ
\end{enumerate}
ことが示されたとき,これを根拠にすべての自然数$n$に対して$P(n)$が成り立つことを
結論する証明法のことである.
さらに,
\begin{enumerate}
  \item $P(1)$が成り立つ.
  \item 各$k$について,$P(1),P(2),\ldots, P(k)$がすべて成り立つならば$P(k+1)$が成り立つ
\end{enumerate}
が示されたとき,これを根拠にすべての自然数$n$に対して$P(n)$が
成り立つことを結論する証明法のことも
数学的帰納法と呼ぶことにする.

%

\section{「推測して帰納法」の問題点}


漸化式によって帰納的に定義された数列の一般項を求める
という問題は,大学受験レベルの問題でよくあるものである.
その中で,「一般項を推測して帰納法で示す」というものがあった.
この解法に関しては,例えば\cite{izumi}において批判的な議論がなされている.
しかし,インターネット上で調べてみた限り,
\cite{izumi}も含めてあまり踏み込んだ議論は
なされてはいないと感じたので,あらためて議論していくことにする.


高校の数学Bの教科書\cite{kyoukasyo}を見てみると,
次のような問題が記載されてる.

\begin{que} \label{que:suisoku}
  次の条件によって定められる数列$\{ a_n \}$がある.
  \begin{align*}
    a_1 & = 2 , \\
    a_{n+1} & = 2- \frac{2}{a_n} \quad ( n = 1,2,3, \ldots).
  \end{align*}
  \begin{enumerate}
    \item $a_2,a_3,a_4$を求めよ.
    \item 第$n$項$a_n$を推測して,それを数学的帰納法を用いて証明せよ.
  \end{enumerate}
\end{que}

教科書ではこの問題は演習問題として載せられているため,
詳細な解答は載っていない.
解答としては「$a_n=(n+1)/n$と推測できる」と
あるのみである.そこで,
出版社が同じであり,かつ
大学受験用参考書として定評のある\cite{chert}中の
類似した問題を見てみる.
\begin{que} \label{que:suisokuchert}
  \begin{align}
    \begin{aligned}
      a_1 & = 1, \\
      a_{n+1} & = \frac{a_n}{1 + 3 a_n} \quad ( n = 1,2,\ldots )
    \end{aligned}
    \label{eq:zenkasiki}
  \end{align}
  で定められる数列$\{ a_n \}$について,
  \begin{enumerate}
    \item $a_2,a_3,a_4$を求めよ.
    \item $a_n$を$n$で表す式を推測し,それを数学的帰納法で証明せよ.
  \end{enumerate}
\end{que}

この問題には詳細な解答が載せられている.
以下に概略を記しておく.

\begin{ans}
  $a_2=1/4, \, a_3 = 1/7 , \, a_4 = 1/10$だから,
  $a_n = 1/(3n-2) \ (n \in \mathbb{N})$と推測される.
  これを帰納法によって示す.
  $a_1=1/(3 \cdot 1 -2) = 1$より
  $n=1$では成り立つ.
  $a_k= 1/(3k-2)$が成り立つと仮定すると,
  \begin{align*}
    a_{k+1} & = \frac{a_k}{1+3a_k} \\
            & = \frac{ \dfrac{1}{3k-2} }{ 1 + 3 \dfrac{1}{3k-2} } \\
            & = \frac{1}{3k+1}
  \end{align*}
  となる.ゆえにすべての自然数$n$に対して$a_n=1/(3n-2)$となる.
\end{ans}

この解答は,いわゆる受験業界では非常によく知られたものである.
また,「数学的帰納法で」という文言を無視すれば,
別解として次のようなものがあることも見逃せない.

\begin{ans2}
  $a_2 = 1/4 , \, a_3 = 1/7 , \, a_4 = 1/10$であるから,
  $a_n = 1/(3n-2) \ (n \in \mathbb{N})$と推測できる.
  実際,$a_n = 1/(3n-2) \ (n \in \mathbb{N})$とすると,
  この数列$\{ a_n \}$は,$a_1 = 1/(3 \cdot 1-2)=1$を
  満たし,さらに各$n \in \mathbb{N}$に対して
  \begin{align*}
    \frac{a_n}{1+ 3 a_n} & = 
    \frac{ \dfrac{1}{3n-2} }{ 1+ 3 \dfrac{1}{3n-2} } \\
    & = \frac{1}{3n+1} \\
    & = a_{n+1}
  \end{align*}
  を満たす.
\end{ans2}


 
まず[別解]に関してだが,明確に穴があるように思われる.
それは,「数列$\{ 1/(3n-2) \}$以外に\prettyref{eq:zenkasiki}を
満たす数列が存在するかどうか」がまったく検討されていない点である.
このギャップを埋めるには,[別解]に続けて\prettyref{eq:zenkasiki}を
満たす任意の数列$\{ a_n \}$が$\{ 1/(3n-2) \}$に等しいことを示せばよい.
すなわち,\prettyref{eq:zenkasiki}を満たす任意の数列$\{a_n \}$に対し,
すべての自然数$n$に対して$a_n = 1/(3n-2)$が成り立つことを示せばよい.
帰納法によるのが普通だろう.
この証明はすでに述べられたような気がしなくもない.
実際に書いてみればわかると思うが,これは
\emph{[解答]で述べた証明とまったく同じである.}


[解答]と[別解]で何をやったのかを整理しよう.
[解答]では,「\prettyref{eq:zenkasiki}を満たす数列$\{ a_n \}$を
任意にとったとき,すべての自然数$n$に対して$a_n = 1/(3n-2)$が成り立つ」
ことを帰納法によって示している.
このことからわかるのは,「\prettyref{eq:zenkasiki}を満たす数列$\{ a_n \}$で,
数列$\{ 1/(3n-2) \}$と等しくないものは存在しない」ということである.
つまり,[解答]では\prettyref{eq:zenkasiki}を満たす数列が
\emph{たかだか}1つであるということを示しているのである.
[別解]では,「数列$\{ 1/(3n-2) \}$が\prettyref{eq:zenkasiki}を満たすこと」
を示している.すなわち,[別解]では
「\prettyref{eq:zenkasiki}」を満たす数列が存在すること
を示しているのである.

従って,\prettyref{eq:zenkasiki}を満たす数列が数列$\{ 1/(3n-2) \}$
\emph{のみである}ことを示すためには,[解答]と[別解]の
両方の記述を合わせなくてはならず,
どちらも問題の解答としては不十分なのである
\footnote{
  事態はもっとややこしい.すぐ後に詳しく解説する.
}
.


\section{何が問題なのか}
「推測して帰納法」という解法の問題点を述べたが,この言説には致命的な欠陥が存在する.
それは,実は,[解答]の中で「数列$\{ 1/(3n-2) \}$が\prettyref{eq:zenkasiki}を
満たすこと」とほぼ同じことが示されているのである.

[解答]の中で,$1/(3 \cdot 1 -2) =1$であり,
$a_k = 1/(3k-2)$と仮定すると$a_{k+1} = 1/(3k+1)$
となることが示されているが,これはまさに「数列$\{ 1/(3n-2) \}$が
\prettyref{eq:zenkasiki}を満たすこと」の証明にほかならない.
となると,[解答]は\prettyref{eq:zenkasiki}を満たす数列の
存在と一意性を同時に示しているのだと解釈できることとなる
\footnote{
  もちろん,[解答]はこのことがクリアにわかるようにはなっていない.
  おそらく,解答作成者は上記のような事情を理解できていないのだろう.
}
.

このことを根拠に,[解答]は問題への解答として
完全なものであると主張するのは,
至極まっとうなものであるように思われる.
しかし筆者が問題視しているのはそこではなく,
あたかも「帰納法によって推測が正しいことが示された」
かのように認識されていることである.

[解答]においては\prettyref{eq:zenkasiki}を満たす
数列の存在と一意性を示していたが,
帰納法を用いたのは一意性の証明のみであったことに
注意しなければならない.
というのも,数列$\{ a_n \}$を$a_n = 1/(3n-2) \ (n \in \mathbb{N})$
と定めたときに,$a_1 = 1$と各$n \in \mathbb{N}$に対して
$a_{n+1} = 1 / (1+ 3 a_n)$が成り立つことは直接的な計算に
よって示すことができる.
ここには帰納法を使うときに出てくるような計算はあるけども,
帰納法自体は使ってはいない.
帰納法を使うのは一意性の方である.
\prettyref{eq:zenkasiki}を満たす数列$\{ a_n \}$を
任意にとったとき,任意の自然数$n$に対して
$a_n = 1/(3n-2)$が成り立つことを示すのだが,
上の計算とまったく同様にして
\begin{enumerate}
  \item $a_1 = 1 ,$
  \item $a_k = 1/(3k-2)$と仮定すると$a_{k+1} = 1/(3k-2)$が成り立つ
\end{enumerate}
ことが示される.ゆえに帰納法によって
任意の自然数$n$に対して$a_n = 1/(3n-2)$が成り立つことを結論しているのである.
この2つの議論において,「数列$\{ a_n \}$」が指す対象が
明確に異なっている.このことを意識しない限り,
「推測して帰納法」の問題点は理解できないだろう.

また,\prettyref{que:suisoku}や\prettyref{que:suisokuchert}の
問題文には「推測が正しいことを帰納法で証明せよ」などとは書かれておらず,
「それを帰納法で証明せよ」と書かれているのみである.
しかし,この「それ」が指す文言が「推測が正しいこと」
であることは疑いようもない.
そしてこの「推測正しいこと」が何を指すかを考えると,
\prettyref{que:suisokuchert}でいえば
「$a_n = 1/(3n-2)$が確かに任意の自然数$n$に対して
\prettyref{eq:zenkasiki}を満たすこと」であろう.
これは推測が間違っていたときのことを考えるとわかりやすい.
例えば,$a_1=1, \, a_{n+1} = a_n + 1 \ (n \in \mathbb{N})$を満たす
数列$\{ a_n \}$について,$a_1=1, \, a_2 = 2, \, a_3 = 3 $であるが,
このことから$a_n = n + (n-1)(n-2)(n-3)$と推測したとすると,
この推測は間違いである.実際,$n=4$のとき,漸化式からは$a_4 = a_3 + 1 = 4$
となるが,$4 + (4-1)(4-2)(4-3) = 10$となり,$n=4$のときには
$a_n = n + (n-1)(n-2)(n-3)$は漸化式を満たさない.


\prettyref{eq:zenkasiki}のような漸化式を満たす数列を考えるとき,
のちに\prettyref{thm:inddef}で示すように,
漸化式を満たすような数列が一意に存在することが知られている.
漸化式に関する問題が軒並み
「次の条件を満たす数列をすべて求めよ」ではなく
「次の条件によって定義される数列の一般項を求めよ」
と書かれているのはそのためである.
このことを「暗黙の了解」とするならば,
[別解]も問題の解答として完全なものになる
\footnote{
  「暗黙の了解」とするのは証明するのが面倒な存在の方ではなく,
  簡単に証明できる一意性の方である.
}
.


上記のような事情から,やはり[解答]や
[別解]の記述は問題ないのではないか
と思われても困るので,
筆者の主張を支持する例として,
\cite{chert}にある問題で,解答に「推測して帰納法」を使っている問題を
もう1問挙げてみよう.
\begin{que} \label{que:suisokuchert2}
  数列$\{ a_n \}$(ただし$a_n > 0$)について次の関係式が
  成り立つとき,一般項$a_n$を推測し,
  その推測が正しいことを証明せよ.
  \begin{align}
    ( a_1 + a_2 + \cdots + a_n )^2 =
    {a_1} ^3 + {a_2} ^3 + \cdots + {a_n } ^3 .
    \label{eq:suisoku2}
  \end{align}
\end{que}

この問題にも解答が載せられている.
\cite{chert}にある解答の概略を述べよう.

\begin{ans}
  関係式において,$a_n > 0$であるから,
  $a_1=1, \, a_2=2, \, a_3=3$である.
  以上から,$a_n = n$と推測できる.
  これを帰納法で示す.
  $a_1 = 1$である.
  $a_1=1, \, \ldots, a_k = k$と仮定すると,
  \begin{align*}
    (1+2+ \cdots + k + a_{k+1})^2 & = 1^3 + 2^3 + \cdots + k^3 + {a_{k+1}} ^3, \\
                                  & \text{(中略)} \\
    a_{k+1}(a_{k+1} + k) (a_{k+1} - k - 1) & = 0.
  \end{align*}
    ゆえに$a_{k+1} > 0$だから$a_{k+1} = k+1$となる.
    従って,任意の自然数$n$に対して$a_n = n$である.
\end{ans}



\prettyref{que:suisokuchert2}の解答では,
文中の条件(\prettyref{eq:suisoku2}と$a_n > 0$)を満たす数列$\{ a_n \}$について,
任意の自然数$n$に対して$a_n = n$が成り立つことを帰納法で示している.
これはつまり,「任意の数列$\{ a_n \}$について,
文中の条件が成り立つならば任意の自然数$n$
に対して$a_n = n$である」
という主張を証明したことになる.
そして,\prettyref{que:suisokuchert2}の[解答]の中には
「数列$\{ n \}$が条件を満たすこと」の証明はどこにもない.
すなわち,\prettyref{que:suisokuchert2}の[解答]では
「推測が正しいこと」は証明されてはいないのである.
にもかかわらず,このような形式の解答は
「推測して帰納法」なる解法として
多くの大学受験用の参考書やウェブサイトで紹介されている.

参考書の答えが間違っているなどということは,
正直このような文章を書いてまで指摘しなければいけないことではない.
もっとも問題視すべきは,
ちょっと考えれば間違いであることはすぐにわかるのに,
「参考書にそう書いてあったから」とか「そう教わったから」
として思考停止してしまっていることである.
数学教員を志している筆者としては,
きちんと批判をしておく必要があると考えている.


\section{教科書・参考書の訂正案}

\prettyref{que:suisokuchert}や\prettyref{que:suisokuchert2}
の解答の問題点を挙げた.そこでは,ほんの少し考えるだけで,
一般に広まっている「推測して帰納法」という解法に
誤りがあることを見出すことができることを述べたのであった.
では,どう訂正するべきかと考えるのは自然なことである.
そこで,筆者は次のような訂正案を提案する.
もう一度問題を提示しておく.

\begin{que} \label{que:suisokuchertkai}
  \begin{align}
    \begin{aligned}
      a_1 & = 1, \\
      a_{n+1} & = \frac{a_n}{1 + 3 a_n} \quad ( n = 1,2,\ldots )
    \end{aligned}
    \label{eq:zenkasikikai}
  \end{align}
  で定められる数列$\{ a_n \}$について,
  \begin{enumerate}
    \item $a_2,a_3,a_4$を求めよ.
    \item $a_n$を$n$で表す式を推測し,それを数学的帰納法で証明せよ.
  \end{enumerate}
\end{que}



\begin{ans} 
  $a_n = 1/(3n-2)$と推定できる.実際,$a_n = 1/(3n-2)$とすると,この$a_n$は
  $a_1= 1/(3 \cdot 1 -2) =1$を満たし,さらに各$n$について
  \begin{align*}
    \frac{a_n}{1+ 3 a_n} & = 
    \frac{ \dfrac{1}{3n-2} }{ 1+ 3 \dfrac{1}{3n-2} } \\
    & = \frac{1}{3n+1} \\
    & = a_{n+1}
  \end{align*}
  を満たす.また,数列$\{ b_n \}$が\prettyref{eq:zenkasikikai}を
  満たすとすると,
  $b_1 = 1 = 1/(3 \cdot 1 -2)$が成り立ち,さらに$b_k=1/(3k-2)$
  と仮定すると,上と同様の計算によって$b_{k+1} = 1/(3k+1)$が成り立つ.
  従って数学的帰納法により,
  すべての自然数$n$に対して$b_n = 1/(3n-2)$が成り立つので,
  \prettyref{eq:zenkasikikai}を満たす数列$\{ a_n \}$は
  $a_n = 1/(3n-2) \ (n =1,2, \ldots)$のみである.
\end{ans}

\begin{que} \label{que:suisokuchert2kai}
  数列$\{ a_n \}$(ただし$a_n > 0$)について次の関係式が
  成り立つとき,一般項$a_n$を推測し,
  その推測が正しいことを証明せよ.
  \begin{align}
    ( a_1 + a_2 + \cdots + a_n )^2 =
    {a_1} ^3 + {a_2} ^3 + \cdots + {a_n } ^3 .
    \label{eq:suisoku2kai}
  \end{align}
\end{que}


\begin{ans}
  $a_n = n \ (n= 1,2, \ldots)$と推測できる.
  実際,$a_n = n$とすると
  \begin{align*}
    \left( \sum_{k=1}^{n} a_k \right) = \left( \frac{1}{2} n(n+1) \right) ^2
    = \sum_{k=1}^{n} {a_k}^3
  \end{align*}
  より\prettyref{eq:suisoku2kai}は成り立ち,さらにつねに$a_n > 0$である.
  また,数列$\{ b_n \}$が
  \prettyref{eq:suisoku2kai}を満たすとすると,
  \prettyref{que:suisokuchert2}の[解答]と同様の議論により,
  $b_1=1$が成り立ち,$b_k=k$と仮定すると$b_{k+1} = k+1$となることがわかる.
  ゆえにすべての自然数$n$に対して$b_n=n$となるので,
  条件を満たす数列は$a_n = n \ (n=1,2,\ldots)$のみである.
\end{ans}

これらの解答では,どのように議論が進んでいるかが明確になっていることがわかるだろう.
また,アヤシイ「暗黙の了解」も一切使っていない.
もともと,一般項が具体的にわかる数列においては,
「漸化式の解の存在と一意性」という仮定は不要である.




\section{学習指導要領はどうなっているか}
以上で「推測して帰納法」という解法の問題点と解決策を述べたわけだが,
国レベルではどう扱われているのかも気になるところである.
国が課す制約として,文部科学大臣から告示される学習指導要領がある.
これは,全国の教育水準を維持するため,
各学校が編成する教育計画(これを教育課程という)の
大綱的な基準を定めたものである\cite{monkasyo}.
この学習指導要領は法的拘束力が有するとされており,
このことからよく「学校教育は中身がガチガチに規定されており,自由度がない」
という言説を見かけることがあるが,学習指導要領には
そのような具体的な方法論や細かい内容に関する記述はない
\footnote{
  だからこそ,新学習指導要領に,いわゆる「アクティブ・ラーニング」を
  盛り込んだことが話題になったのである.
}
.

名前の似ている文書として,学習指導要領解説というものがある.
これは文部科学省の著作物で,
学習指導要領等の改善の趣旨や内容を解説したものである\cite{monkasyo}.
こちらにはかなり具体的な内容や方法論が書かれているが,
法的拘束力はなく,無理に従う必要はない.


さて,現行の高校の学習指導要領\cite{youryou}の数列のところを
見てみると,当然のことながら「推測して帰納法」に明確に関連すると思われる記述はない.
しかし,学習指導要領解説\cite{youryoukai}には次のように記述されている.

\begin{quote}
  漸化式を用いて表される数列の一般項を推測し,
  数学的帰納法を用いてその推測が正しいことを証明することも考えられる。
  例えば,$a_{n+1} = 3 a_n + 2 , \, a_1=1$で表される数列の一般項を
  $2 \times 3^{n-1} -1 $と推測して証明することなどを扱う。
\end{quote}

明らかに「推測して帰納法」を取り上げさせる意図が見えている.
もちろん,どのように証明するのかに関する詳細は述べられてはいないが,
「問題がある解法」として紹介した方の解法が
想定されているであろうことは想像に難くない.
高校生向けの教科書併用問題集や参考書のほとんどに
誤りがあるという認識があるならば,学習指導要領解説において
何らかの記述があるはずだからである.
このことは,新学習指導指導要領解説においても変わっていない.


現状では,一般に指導者とされる人間の多くに
先に述べたような誤解が広まってしまっているため,
この誤解がまた次の世代まで引き継がれてしまう可能性は高い.
誰かがこの誤解の連鎖を断ち切らなければならない.
本稿が,そのきっかけの1つとなれば幸いである.




\section{帰納的定義について}


以下,$0$は自然数に含まれるとする.
また,集合と写像に関する基本的な知識は既知とする.
さらに,以下の点に注意していただきたい.
\begin{itemize}
  \item 写像$f \colon X \longrightarrow Y$と$A \subset X$に対し,
    写像$f|_A \colon A \longrightarrow Y$は$f$の始集合を$A$に
    制限したものを表す.
  \item 整列集合$(W, \leq)$と$x \in W$に対し,$W$の部分集合
    $\Set{ w \in W \mid w < x}$を$W \langle x \rangle$と表す.
    整列集合について知らなくても,ひとまず$n \in \mathbb{N}$について,
    $\mathbb{N} \langle n \rangle = \Set{ 0,1, \ldots n-1}$であることが
    わかれば問題ない.
    $W = \mathbb{N} - \Set{0}$としたときに,$n \in W$に対して
    $W \langle n \rangle = \Set{1,2, \ldots, n-1}$となることも
    すぐにわかるだろう.
\end{itemize}


漸化式により数列が帰納的に定義できることは,
主張自体はよく見かけるものであるが,その証明は
ちょっとググった程度では出てきてはくれない.
もちろんちゃんとした本を見れば載っているのだが,
せっかくの機会なのでここで定式化して証明まで済ませてしまおう.

\begin{thm} \label{thm:inddef}
  $W= \mathbb{N} - \Set{0}$とし,
  空でない集合$X$と写像$G \colon \bigcup_{n \in W } 
  X^{W \langle n \rangle} \longrightarrow X$
  が与えられたとする.
  このとき,$x_0 \in X$を1つ定めるごとに,
  写像$f \colon \mathbb{N} \longrightarrow X$で
  \begin{align}
    \begin{aligned}
      f(0) & = x_0 , \\
      f(n) & = G \left( f|_{W \langle n \rangle} \right)
      \quad ( n \in W )
    \end{aligned}
    \label{eq:inddef}
  \end{align}
  を満たすものが一意に定まる.
\end{thm}

\begin{proof}
  先に一意性の方から示しておこう.
  写像$f,g \colon \mathbb{N} \longrightarrow X$で
  \prettyref{eq:inddef}を満たすようなものを任意にとり,
  すべての$n \in \mathbb{N}$に対して$f(n) = g(n)$が成り立つことを
  帰納法で示す.
  $f(0) = x_0 = g(0)$より$f(0)=g(0)$である.
  各$n \in \mathbb{N}$に対し,$i= 0,1, \ldots, n$に対して
  $f(i)=g(i)$であるとすると,
  $f|_{W \langle n+1 \rangle} = g|_{W \langle n+1 \rangle} $
  だから$f(n+1) = G \left( f|_{W \langle n+1 \rangle} \right) =
  G \left ( g |_{W \langle n+1 \rangle} \right) = g(n+1)$
  より$f(n+1) =g(n+1)$となる.
  ゆえにすべての$n \in \mathbb{N}$に対して$f(n) = g(n)$となり,
  $f=g$を得る.

  \prettyref{eq:inddef}を満たす写像の存在を示そう.
  $n \in W$に関する条件$P(n)$を
  「写像$f \colon W \langle n \rangle \longrightarrow X$で
  すべての$m \in W \langle n \rangle$に対して
  $f(m) = G \left( f|_{W \langle m \rangle } \right)$
  を満たすものが存在する」と定める.
  一意性の証明とまったく同様にして,
  各$n \in W$に対して$P(n)$を成り立たせるような写像$f$は
  たかだか1つであることが示される.その写像を$f_n$と表記する.
  すべての$n \in W$に対して$P(n)$が成り立つことを帰納法によって示そう.
  $W \langle 1 \rangle = \varnothing$だから,
  $P(1)$を成り立たせるような写像$f$として空写像がとれる.従って$P(1)$は成り立つ.
  各$n \in W$に対し,$i=1,2,\ldots,n$に対して
  $P(i)$が成り立つと仮定する.
  いま,写像$f \colon W \langle n+1 \rangle \longrightarrow X$を
  \begin{align*}
    f (m) = 
    \begin{cases}
      G \left( f_{m} \right) & ( \text{$m=n$のとき} ) , \\
      f_{m+1} (m) & (\text{それ以外のとき})
    \end{cases}
  \end{align*}
  と定めると,この$f$はすべての$m \in W \langle n+1 \rangle$
  に対して$f(m) = G \left( f|_{W \langle m \rangle} \right)$
  を満たす.ゆえに$P(n+1)$も成り立つから,
  すべての$n \in W$に対して$P(n)$が成り立つ.
  そこで,あらためて写像$f \colon \mathbb{N} \longrightarrow X$を
  \begin{align*}
    f(0) & = x_0 , \\ 
    f(n) & = f_{n+1} (n) \quad ( n \in W)
  \end{align*}
  と定めると,この$f$は\prettyref{eq:inddef}を満たす.
\end{proof}


$W = \mathbb{N} - \Set{0}$とし,
空でない集合$X$と写像$g \colon X \times \mathbb{N} \longrightarrow X$が与えられたとする.
各$\varPhi \in \bigcup _{n \in W} X^{W \langle n \rangle}$に対し,
$\varPhi \in X^{W \langle i \rangle}$となる$i \in W$がただ1つ存在する.
$\varPhi (i-1) \in X $だから,
$g( \varPhi (i-1) , i-1) \in X$である.
$\varPhi \in \bigcup_{n \in W} X^{W \langle n \rangle }$を上記の
$g (\varPhi (i-1) , i-1) \in X$に対応させる写像を$G$とすると,
\prettyref{thm:inddef}より,
$x_0 \in X$を1つ定めるごとに,写像$f \colon \mathbb{N} \longrightarrow X$で
\begin{align*}
  f(0) & = x_0 \\
  f(n) & = G \left( f|_{W \langle n \rangle} \right) \quad (n \in W)
\end{align*}
を満たすものがただ1つ定まる.
各$n \in \mathbb{N}$に対し,$G \left( f|_{W \langle n+1 \rangle} \right) 
= g (f(n), n)$だから,
写像$f \colon \mathbb{N} \longrightarrow X$で
\begin{align}
  \begin{aligned}
    f(0) & = x_0 \\
    f(n+1) & = g(f(n),n) \quad (n \in \mathbb{N})
  \end{aligned}
  \label{eq:zenkasikimap}
\end{align}
を満たすものがただ1つ定まることになる.
$X$を実数全体の集合$\mathbb{R}$とし,
$\mathbb{N}$から$\mathbb{R}$への写像が数列だと解釈すれば,
これで漸化式による数列の帰納的定義が保証されたことになる.
また,\prettyref{thm:inddef}においては,隣接2項間の漸化式に
関してしか言及できてはいないが,隣接3項間,4項間といった漸化式においても
同様であることは明らかであろう.

%
\printbibliography[title=参考文献]
\end{document}
